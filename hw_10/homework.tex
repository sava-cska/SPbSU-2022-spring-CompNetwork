\documentclass[12pt]{article}

\usepackage{complexity}
\usepackage{cmap}
\usepackage[T2A]{fontenc}
\usepackage[utf8]{inputenc}
\usepackage[russian]{babel}
\usepackage{graphicx}
\usepackage{amsthm,amsmath,amssymb}
\usepackage[russian,colorlinks=true,urlcolor=red,linkcolor=blue]{hyperref}
\usepackage{enumerate}
\usepackage{datetime}
\usepackage{minted}
\usepackage{fancyhdr}
\usepackage{lastpage}
\usepackage{color}
\usepackage{verbatim}
\usepackage{tikz}
\usepackage{epstopdf}
\usepackage{enumitem}

\def\THEME{Домашнее задание 10}
\newcommand{\PrE}{\mathbb{E}}
\newcommand{\PrD}{\mathbb{D}}
\newcommand{\PrP}{\mathbb{P}}

\begin{document}

\begin{center}
\vspace*{0mm}
{\LARGE \bf \THEME}
\end{center}

\begin{center}
{\Large \it Wireshark. IP}
\end{center}

\begin{enumerate}

\item Мой \texttt{IP}-адрес~--- \texttt{192.168.0.171}.

\item Указан протокол \texttt{ICMP}.

\item В \texttt{IP}-заголовке 20 байт, а полезная нагрузка составляет $56 - 20 = 36$ байт.

\item

\begin{enumerate}

\item У последовательных дейтаграмм в \texttt{IP}-заголовках различаются поля идентификатора, TTL и контрольной суммы.

\item Все остальные поля остаются неизменными. Поскольку мы отсылаем один и тот же пакет с разным TTL, то так и должно быть, что у дейтаграмм все поля, кроме идентификатора, TTL и контрольной суммы, остаются неизменными, а отмеченные поля меняются.

\item Это поле увеличивается на 1 с каждой последующей дейтаграммой.

\end{enumerate}

\end{enumerate}

Скрины в папке \texttt{wireshark\_ip}.


\end{document}