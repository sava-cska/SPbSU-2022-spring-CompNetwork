\documentclass[12pt]{article}

\usepackage{complexity}
\usepackage{cmap}
\usepackage[T2A]{fontenc}
\usepackage[utf8]{inputenc}
\usepackage[russian]{babel}
\usepackage{graphicx}
\usepackage{amsthm,amsmath,amssymb}
\usepackage[russian,colorlinks=true,urlcolor=red,linkcolor=blue]{hyperref}
\usepackage{enumerate}
\usepackage{datetime}
\usepackage{minted}
\usepackage{fancyhdr}
\usepackage{lastpage}
\usepackage{color}
\usepackage{verbatim}
\usepackage{tikz}
\usepackage{epstopdf}
\usepackage{enumitem}

\def\THEME{Домашнее задание 13}
\newcommand{\PrE}{\mathbb{E}}
\newcommand{\PrD}{\mathbb{D}}
\newcommand{\PrP}{\mathbb{P}}

\begin{document}

\begin{center}
\vspace*{0mm}
{\LARGE \bf \THEME}
\end{center}

\begin{center}
{\Large \it Wireshark. DHCP}
\end{center}

\begin{enumerate}

\item Сообщения \texttt{DHCP} посылаются поверх протокола \texttt{UDP}.

\item Адрес канального уровня моего компьютера~--- \texttt{e4:aa:ea:63:26:c5}.

\item Значение \texttt{TransactionID} равняется \texttt{0xa490d293}. Это число нужно, чтобы верно идентифицировать все сообщения одного процесса (например, как в этом случае, процесса получения \texttt{ip}-адреса), поскольку во всех сообщениях \texttt{TransactionID} совпадает.

\item \texttt{ip}-адрес источника в дейтаграмме, отправленной с хоста,~--- $0.0.0.0$, а \texttt{ip}-адрес адресата~--- $255.255.255.255$.

\item \texttt{ip}-адрес \texttt{DHCP}-сервера~--- $192.168.0.1$.

\item \texttt{DHCP}-сервер выдаёт \texttt{ip}-адрес хосту на некоторое время, называемое \textit{временем аренды}, а по его истечении хост должен запросить новый адрес, который, возможно, будет совпадать с предыдущим. Это делается, чтобы один \texttt{ip}-адрес не закреплялся навечно за некоторым хостом. В текущем задании срок аренды~--- 1 день.

\end{enumerate}

Скрины в папке \texttt{wireshark\_dhcp}.

\end{document}