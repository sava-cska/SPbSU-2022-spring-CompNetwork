\documentclass[12pt]{article}

\usepackage{complexity}
\usepackage{cmap}
\usepackage[T2A]{fontenc}
\usepackage[utf8]{inputenc}
\usepackage[russian]{babel}
\usepackage{graphicx}
\usepackage{amsthm,amsmath,amssymb}
\usepackage[russian,colorlinks=true,urlcolor=red,linkcolor=blue]{hyperref}
\usepackage{enumerate}
\usepackage{datetime}
\usepackage{minted}
\usepackage{fancyhdr}
\usepackage{lastpage}
\usepackage{color}
\usepackage{verbatim}
\usepackage{tikz}
\usepackage{epstopdf}
\usepackage{enumitem}

\def\THEME{Домашнее задание 14}
\newcommand{\PrE}{\mathbb{E}}
\newcommand{\PrD}{\mathbb{D}}
\newcommand{\PrP}{\mathbb{P}}

\begin{document}

\begin{center}
\vspace*{0mm}
{\LARGE \bf \THEME}
\end{center}

\begin{center}
{\Large \it Wireshark. Ethernet}
\end{center}

\begin{enumerate}

\item \texttt{Ethernet}-адрес моего компьютера~--- \texttt{e4:aa:ea:63:26:c5}.

\item \texttt{LiteonTe\_63:26:c5}~--- мой компьютер.

\item Исходный адрес~--- \texttt{b0:be:76:1b:79:0e}. Этот адрес соответствует устройству \texttt{Tp-LinkT\_1b:79:0e}~--- роутеру.

\item Адрес назначения~--- \texttt{e4:aa:ea:63:26:c5}. Это адрес моего компьютера.

\end{enumerate}

Скрины в папке \texttt{wireshark\_ethernet}.

\begin{center}
{\Large \it Wireshark. ARP}
\end{center}

\begin{enumerate}

\item Исходный адрес~--- \texttt{e4:aa:ea:63:26:c5}, конечный адрес~--- \texttt{ff:ff:ff:ff:ff:ff}.

\item Да, в запросе указан \texttt{ip}-адрес моего компьюьтера \texttt{192.168.0.113}, а в ответе~--- \texttt{ip}-адрес роутера \texttt{192.168.0.1}.

\item \texttt{Target MAC address: 00:00:00\_00:00:00} в сообщении-запросе.

\item \texttt{Sender MAC address: Tp-LinkT\_1b:79:0e} в сообщении-ответе.

\end{enumerate}

Скрины в папке \texttt{wireshark\_arp}.

\end{document}